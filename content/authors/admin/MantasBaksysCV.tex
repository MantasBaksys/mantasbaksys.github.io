%%%%%%%%%%%%%%%%%%%%%%%%%%%%%%%%%%%%%%%%%
% Medium Length Professional CV
% LaTeX Template
% Version 2.0 (8/5/13)
%
% This template has been downloaded from:
% http://www.LaTeXTemplates.com
%
% Original author:
% Trey Hunner (http://www.treyhunner.com/)
%
% Important note:
% This template requires the resume.cls file to be in the same directory as the
% .tex file. The resume.cls file provides the resume style used for structuring the
% document.
%
%%%%%%%%%%%%%%%%%%%%%%%%%%%%%%%%%%%%%%%%%

%----------------------------------------------------------------------------------------
%	PACKAGES AND OTHER DOCUMENT CONFIGURATIONS
%----------------------------------------------------------------------------------------

\documentclass{resume} % Use the custom resume.cls style

\usepackage[left=0.75in,top=0.6in,right=0.75in,bottom=0.6in]{geometry} 
\usepackage[parfill]{parskip}
\usepackage{hyperref}% Document margins
\usepackage{enumitem}


\renewcommand{\refname}{}

\name{Mantas Bakšys} % Your name
\address{mantasbaksys.github.io \\ mb2412@cam.ac.uk \\ github.com/MantasBaksys} % Your phone number and email


\begin{document}


%----------------------------------------------------------------------------------------
%	WORK EXPERIENCE SECTION
%----------------------------------------------------------------------------------------


%------------------------------------------------

\begin{rSection}{Experience}

\begin{rSubsection}{AWS Reasoning Group}{July 2025 - Present}{Applied Scientist Intern}{Full-time, London, UK}

\item Supervised by Dr. Stefan Zetsche, working on RL for program verification.

\end{rSubsection}

\begin{rSubsection}{Project Numina}{September 2024 - Present}{Researcher}{Part-time, Remote}
\item Using RL techniques to train AI models for Interactive Theorem Proving in Lean.
\item Development of Lean-Python interfaces for dataset and benchmark creation and curation.
\item Released open-source models for autoformalization and proving in Lean and achieved state-of-the art performance on the miniF2F benchmark in two separate realeases (\href{https://arxiv.org/abs/2504.11354}{arXiv}, \href{https://huggingface.co/blog/AI-MO/kimina-prover}{HuggingFace}).

\end{rSubsection}


\begin{rSubsection}{University of Cambridge, DPMMS}{July 2023 - September 2023}{Summer Research Student}{Cambridge, UK}
\item Worked in a research group led by Professor Timothy Gowers on automated theorem-proving.
\item Focus on motivated proof discovery using the Lean theorem prover.
\item Jointly developed an interactive Lean program to guide further research in human-oriented proving.
\end{rSubsection}

%------------------------------------------------

\begin{rSubsection}{University of Cambridge, Computer Lab}{June 2022 - August 2022}{Summer Research Student}{Cambridge, UK}
\item Worked in the ALEXANDRIA group supervised by Dr. Angeliki Koutsoukou-Argyraki.
\item Successful formalisation of Master's level material in Additive Combinatorics including \\ the Balog-Szemeredi-Gowers theorem.
\item Co-authored a paper accepted to CPP $2023$ (\href{https://dl.acm.org/doi/10.1145/3573105.3575680}{Link to Open Access}).

\end{rSubsection}

\begin{rSubsection}{Open AI}{December 2021 - January 2022}{Research Intern}{Online}
\item Worked with Stanislas Polu on training ML models to find formal proofs using the Lean theorem prover.
\item Analyzed properties of trained models and investigated new research directions.
\item Co-authored a paper accepted to ICLR $2023$, which can be found on \href{https://arxiv.org/abs/2202.01344}{arXiv}.
\end{rSubsection}


%------------------------------------------------

\begin{rSubsection}{University of Cambridge, DPMMS}{June 2021 - August 2021}{Summer Research Student}{Cambridge, UK}
\item Worked with Dr Aled Walker investigating  the Multiplication Table problem for bipartite graphs.
\item Co-authored a preprint that can be found on \href{https://arxiv.org/abs/2109.08485}{arXiv}.
\end{rSubsection}

%------------------------------------------------

%\begin{rSubsection}{Visma}{June 2019 - July 2019}{Software Development Intern}{Vilnius, Lithuania}
%\item Built an AutoChess based combat simulator using C\#
%\item Developed a program to predict outcomes of a combat simulator using regression trees
%\end{rSubsection}

\end{rSection}

%----------------------------------------------------------------------------------------
%	EDUCATION SECTION
%----------------------------------------------------------------------------------------
\begin{rSection}{Education}

{\bf St. Catharine's College, University of Cambridge} \hfill {2024-2028 (expected)} \\
PhD, Computer Science

\vspace{-0.1cm}

{\bf St. Catharine's College, University of Cambridge} \hfill {2023-2024} \\
MMath, Part III Mathematics \\
Grade: Merit


\vspace{-0.1cm}
{\bf St. Catharine's College, University of Cambridge} \hfill {2020-2023} \\ 
BA (Hons), Mathematics Tripos \\
Grade: 2.i

\vspace{-0.1cm}
{\bf Vilnius Jesuit Gymnasium} \hfill {2016-2020} \\
Lithuanian Brandos Atestatas \\
Overall average grade 10/10 (highest), top 0.5\% %0.5\% in Lithuania
\end{rSection}

%----------------------------------------------------------------------------------------
%	TECHNICAL STRENGTHS SECTION
%----------------------------------------------------------------------------------------
\begin{rSection}{Awards, Achievements \& Hobbies}
\begin{list}{$\cdot$}{\leftmargin=0em}
\setlength\itemsep{-1.5em}
    \item Represented Team Lithuania at the International Mathematical Olympiad \\
    \item Contributor to Lean's mathematical library Mathlib
\end{list}
\end{rSection}
\begin{rSection}{Technical Strengths}

\begin{tabular}{ @{} >{\bfseries}l @{\hspace{6ex}} l }
Computer Languages & Lean, Isabelle/HOL, LaTeX, C++ , C\#, Python, TypeScript, JavaScript 
\end{tabular}

\begin{rSection}{Publication List}
\end{rSection}
\begin{thebibliography}{99}
\vspace{-3.5em}

\bibitem{polu2022formal}
S.~Polu, J.M~Han, K.~Zheng, M.~Baksys, I.~Babuschkin, and I.~Sutskever. \\
\newblock Formal Mathematics Statement Curriculum learning.
\newblock {\em arXiv preprint arXiv:2202.01344, published at ICLR 2023}, 2022.

\bibitem{wang2025kimina-preview}
H.~Wang, M.~Unsal, X.~Lin, M.~Baksys, J.~Liu, M.D~Santos, F.~Sung, M.~Vinyes, et~al. \\
\newblock Kimina-Prover Preview: Towards Large Formal Reasoning Models with Reinforcement Learning.
\newblock {\em arXiv preprint arXiv:2504.11354}, 2025.

\bibitem{wang2025kimina}
H.~Wang, M.~Unsal, X.~Lin, M.~Baksys, J.~Liu, M.D~Santos, F.~Sung, M.~Vinyes, et~al. \\
\newblock Kimina-Prover: Applying Test-time RL Search on Large Formal Reasoning Models
\newblock {\em Huggingface Blog: https://huggingface.co/blog/AI-MO/kimina-prover}, 2025.

\bibitem{koutsoukou2023formalisation}
A.~Koutsoukou-Argyraki, M.~Bakšys, and C.~Edmonds. \\
\newblock A formalisation of the Balog–Szemerédi–Gowers theorem in Isabelle/HOL.
\newblock In {\em Proceedings of the 12th ACM SIGPLAN International Conference on Certified Programs and Proofs}, 2023.


\bibitem{santos2025kimina}
M.D~Santos, H.~Wang, H.~de~Saxcé, R.~Wang, M.~Baksys, M.~Unsal, J.~Liu, et~al. \\
\newblock Kimina Lean Server: Technical Report.
\newblock {\em arXiv preprint arXiv:2504.21230}, 2025.


\bibitem{baksys2025generalization}
M.~Bakšys. \\
\newblock A generalization of the Cauchy–Davenport theorem.
\newblock {\em Archive of Formal Proofs}, 2023.


\bibitem{baksys2022kneser}
M.~Bakšys and A.~Koutsoukou-Argyraki. \\
\newblock Kneser's theorem and the Cauchy–Davenport theorem.
\newblock {\em Archive of Formal Proofs}, 2022.

\end{thebibliography}
\end{rSection}

%----------------------------------------------------------------------------------------
%	EXAMPLE SECTION
%----------------------------------------------------------------------------------------


%----------------------------------------------------------------------------------------

\end{document}
